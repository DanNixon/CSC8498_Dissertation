\documentclass{entcs}
\usepackage{CSC8498macro}

\makeatletter

\def\lastname{CSC8498}

% http://tex.stackexchange.com/questions/283141/latex-error-command-cauthor-already-defined
\makeatletter
\let\c@author\relax
\makeatother

\usepackage[backend=bibtex]{biblatex}
\addbibresource{CSC8498_InterimReport_DanNixon.bib}

\ifx\pdftexversion\undefined
\usepackage[dvips]{graphicx}
\else
\usepackage[pdftex]{graphicx}
\DeclareGraphicsRule{*}{mps}{*}{}
\fi

\begin{document}

\begin{frontmatter}
  \title{Cost effective, large scale, fixed world 3D scanning}
  \author{Dan Nixon}
  \address{School of Computing Science, Newcastle University, UK}
  \thanks[email]{Email:
    \href{mailto:d.nixon2@ncl.ac.uk}
         {\texttt{\normalshape{d.nixon2@ncl.ac.uk}}}}

  % \begin{abstract}
  % \end{abstract}

  \begin{keyword}
    3D scanning, point clouds
  \end{keyword}
\end{frontmatter}

\section{Introduction}
\label{sec:introduction}

This project focuses on the capture of 3D point clouds and the process of
aligning a large number of them to recreate a large environment.

The most common methods for performing this type of environment scan typically
use the point cloud alone to inform the alignment which can add additional
computation cost and in the case where the environment contains very few or
repetitive features, can increase alignment error.

It is hypothesised that if the position and orientation of the camera can be
accurately measured at the same time as the point clouds are captured then the
complexity of the alignment proces can be significantly reduced.

\section{Aim and objectives}
\label{sec:aim_and_objectives}

The aim of this project is to evaluate the performance of such a 3D capture
system in a variety of environments.

\subsection{Objectives}
\label{sec:objectives}

\begin{itemize}
  \item
    Implementation of a point cloud capture program with plugin-style processing
    options.

  \item
    Sense camera position and orientation to a sufficient accuracy without
    camera feedback

  \item
    Implementation of point cloud alignment algorithms informed by camera
    position and orientation

  \item
    Evaluation of the performance (both computational and functional) of
    implemented algorithms

\end{itemize}

\section{Work plan}
\label{sec:work_plan}

The following describes a series of high level tasks that together address the
objectives of the project.

The deliverables for all but the evaluation will be in the form of source code
implementing the relevant features. The evaluation will be presented in the
dissertation.

Due dates for specific tasks are given in brackets.

\subsection{Framework (complete)}
\label{sec:framework}

This consists of the essential framework that will be used to create the capture
and processing applications required for the project and integration of the
required external libraries (Point Cloud Library \cite{Rusu_ICRA2011_PCL},
Eigen3 \cite{eigenweb}, Boost \cite{BoostLibs}) into a build system (CMake
\cite{CMake}).

\subsection{Position and orientation tracking (due 2nd May)}
\label{sec:position_and_orientation_tracking}

The position and orientation of the camera, which in the case of this project
will be a Microsoft Kinect, will be monitored by a device fixed to the camera
body consisting of a series of sensors and a microcontroller performing data
acquisition and sensor fusion as well as communicating with the capture program
running on a PC.

At the very minimum this sensor must measure the orientation using a triple axis
accelerometer and gyroscope and displacement through double integration of 3D
acceleration.

From this baseline further sensors can be added that can be used to both
increase the accuracy of the measurement and provide an absolute measurement.

\subsection{Alignment techniques}
\label{sec:alignment_techniques}

Several processing techniques will be implemented that align a series of
captured point clouds into a single cloud depicting the environment being
scanned.

In each case the alignment technique will be a Normalised Distributions
Transformation (NDT) alignment \cite{merten2008three}. This method describes a
point cloud as a selection of cells where the surface geometry is described by a
probability distribution function.

This representation allows the alignment to be decomposed to an optimisation
problem that aims to maximise the likelihood of overlapping cells corresponding
to the same surface.

\subsubsection{Alignment into world (complete)}

This method performs an NDT alignment of each captured point cloud onto a single
cloud that represents the entire environment scanned.

This allows for the camera position to be changed by any amount as long as there
is still a partial overlap with already scanned environment.

\subsubsection{Incremental capture alignment (complete)}

This method performs an NDT alignment of each new point cloud onto the
previously captured point cloud and saves pairs of point clouds and
transformations from which a world cloud can be created as a post processing
step.

This method requires that each new point cloud captured overlaps the last cloud
captured, however will have a much smaller memory footprint than the world
alignment method mentioned previously.

\subsubsection{Incremental chunk alignment (due 5th May)}

A hybrid of the previous two methods can be implemented where chunks of the
environment are scanned at a time using the first method and then saved once a
certain level of detail is reached along with a transformation such that the
chunks can later be combined to form a much larger point cloud, as per the
second method.

\subsection{Evaluation and Documentation (due 19th May)}
\label{sec:evaluation}

An evaluation of the effectiveness of the three implemented alignment strategies
will be carried out; this will cover environments that pose potential issues to
such a scanning system (i.e. those with similar surfaces or high noise) and a
range of motion behaviours to test the durability of the orientation and
position measurement (e.g. fast/sharp movements are more likely to cause error
in IMU positioning).

This evaluation may be extended to comparison to existing scanning systems,
however the more relevant measure is how well the reconstructed geometry
represents the original.

The implementation of the system will be documented in the dissertation along
with possible issues and improvements.

\printbibliography

\end{document}
